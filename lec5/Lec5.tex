%  Lecture 5: Filters
 Filtering is used to dampen, or $atteutate$, signals of certain frequencies.
They're used in a variety of applications and come in many different types. To
describe the different filters characteristics, certain critera are examined.
Filtering might be needed to filter out noise from quantization(resolution),
mechanical noise and noise from electric disturbanses.
\subsection*{Filter characteristics}
Performance of a filter can be viewed from two different domains, the $frequency
domain$ and the $time domain$. These are interconnected and are both used,
though for general description of a filter, most often the frequency domain is
used.
\subsubsection*{Frequency domain characteristics}
The frequency domain contains information on how the signal through a filter is
attenuated at different frequencies, as well as how the singals phase is
shifted, called $phase shift$. Most commonly, the frequency domain is reached by
$Laplace transformation$ to the variable $s$, though the $Fourier transform$
might also be used. $s$ represents the the complex frequency $j\omega$. The
cutoff requency $f_c$ is described as the frequency at which the signal has
dropped by 3dB. Before $f_c$ is the $passband$ and after comes the $transition
region$ which stops at the $stop band$, defined by some minimum attenuarion, for
example 40 dB. The earlier described $phase shift$ is important because when
different frequencies of a signal don't have the same phase shift, the waveform
might come out distorted.
\subsubsection*{Time domain characteristics}
The time domain describes the behaviour of the output signal when different
functions such as step and ramp are run through the filter. These include $rise
time$, the time it takes for the signal to go between 10\% and 90\% of its
maximum value. The time before the signal settles indefinitely within a
predefined limit, often 5\%, is called $settling time$. The delay before the
signal reaches 10\% of its max is called $time delat$ and the $overshoot$
indicates the absolute maximum value the signal reaches relatively to the steady
state value.
\subsubsection*{Discrete vs continuous filters}
Describing a filter in continuous time is done using Laplace transformation. 
